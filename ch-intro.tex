%!TEX root =  main.tex
%!TEX encoding = UTF-8 Unicode
\chapter{推薦システム}
\label{chap:intro}

最初に『\term{推薦システム}{recommender system}』とは何であるかということについて,その原点の一つであるACM Communications誌での特集\cite{macm:97:01}での記述を紹介する:\singlespacing
\begin{center}
\cornersize*{2zw}\setlength{\fboxsep}{1zw}
\ovalbox{\begin{minipage}{0.92\linewidth}
\itshape%
It is often necessary to make choices without sufficient personal experience of the alternatives.
In everyday life, we rely on recommendations from other people either by word of mouth, recommendation letters, movie and book reviews printed in newspapers, or general surveys such as Zagat's restaurant guides.

Recommender systems assist and augment this natural social process.

\bigskip\small\rmfamily
自分の経験だけでは違いがあまりよくわからないものの中からでも,どうしてもどれかを選ばなければならないということはよくある.
こうしたときには,口コミ,推薦状,新聞の書評や映画評,ザガットのレストランガイドなどの他人からの推薦に頼ることを日常的に行っている.

推薦システムは,こうした社会で普通に行われている一連の行為を補助したり,促進したりする.
\end{minipage}}
\end{center}\medskip
より簡潔には,Konstanによるチュートリアル\cite{sigchi:03:01}の定義がよいだろう:
\begin{center}%
\cornersize*{2zw}\setlength{\fboxsep}{1zw}\singlespacing%
\ovalbox{\begin{minipage}{0.92\linewidth}%
\centering
\itshape
Recommenders: Tools to help identify worthwhile stuff

\medskip\small\rmfamily
推薦システム:どれに価値があるかを特定するのを助ける道具
\end{minipage}}
\end{center}\onehalfspacing
このように,利用者にとって有用と思われる対象,情報,または商品などを選び出し,それを利用者の目的に合わせた形で提示するシステムといえる.

この推薦システムが必要になった背景は大きく二つある.
第一に,大量の情報が発信されるようになったことがある.
これは,情報化技術の進展により,個人・団体が容易かつ低コストで発信できるようになったためである.
第二の理由は,これら大量の情報の蓄積や流通が容易になり,誰もが大量の情報を得ることができるようになったことである.
これも計算機の記憶媒体の大規模化や,通信の高速化によるものである.
これらの要因により,大量に発信された情報を,だれもが大量に取得できる状況が生じた.
しかし,どのように欲しい情報を特定する方法が分からない(例:統計資料として公開されているがその名前が分からない)とか,探している情報を特定できない(例:類似した資料が大量にあり,その中に目的のものが埋もれてしまっている)といった理由により,情報を参照できる状態にあるにもかかわらず,それを識別できないという状況が生じた.
この状況を『\term{情報過多}{information overload}』~\cite{misc:009}(情報爆発 (information explosion) や情報洪水 (information overflow) )という.
この状況に対処するため,利用者にとって有用な情報を見つけ出す推薦システムは考案された.

この推薦システムがどのように誕生し,広まっていったかを述べておく.
広義には情報検索や情報フィルタリング技術の一つと見なせるので,初期の推薦システムはこれらの技術を基盤としていた.
この推薦システムの実現手法の一つに協調フィルタリングがあるが,この用語の方が推薦システムという用語より古く,1992年に文献 \cite{macm:92:01} にて使われた.
しかし,これは現在のような協調フィルタリングではなく,他人が手動で行った推薦を検索できる協調作業支援のシステムであった.
この過程を自動化したシステムが1994年のGroupLens~\cite{cscw:94:01}やRingo~\cite{sigchi:95:02}であり,現在の推薦システムの基礎となった.
\index{GroupLens法}\index{GroupLens method}
同時に,もう一つの実現手法である内容ベースフィルタリングも,従来からある情報フィルタリングとして,また,事例ベース推論の応用としても研究されてきたが,推薦システムとして独自の側面が徐々に強くなっていった.
1996年には,専門のワークショップも開催されるほどに研究が活発化した.
1997年には上記のACM Communications誌での特集\cite{macm:97:01} により,この種のシステムの呼び名として ``recommender system''が定着した.
また,このころには NetPerceptions や Firefly などの企業によってシステムの商業化も始まった.
Webを通じた各種サービスの機能で活用されたり\cite{ieeem:99:02,ieeem:03:01,www:07:01},セットトップボックスなど
の機器に組み込まれたり\cite{kdd:04:11}している.
2000年代以降は,物理的な店舗面積に商品数が制限されない電子商取引の発展や,大量の画一的な商品から,少量多品種を扱う mass cusomization \index{mass customization}への消費傾向の変化に伴って,その重要性も広く認識されるようになった.
このことを象徴する Amazon.com CEOのJeff Bezosの発言を引用しておこう\cite{dmkd:01:01}%
\footnote{J.~Riedlのメールによれば,J.~Bezosは,この発言を幾つかの講演で行った.ここでは,300万人と書いたが,そのときどきの顧客数に応じて,この数字は変えて用いられた.} .
\begin{center}
\cornersize*{2zw}\setlength{\fboxsep}{1zw}
\ovalbox{\begin{minipage}{0.92\linewidth}
\large\itshape
If I have 3 million customers on the Web,\\
\hfill{}I should have 3 million stores on the Web

\small\rmfamily\centering
Webに3百万人の顧客がいるなら,3百万のWebストアを用意すべきだ
\end{minipage}}
\end{center}
現在では,推薦システムは多方面で利用されるようになり,研究も継続的に行われ,多様な方法が目的に応じて考案されている.

この推薦システムには大きく三つの要素技術が関連している.
一つ目は,人間から必要な情報を収集し,との対話を扱うヒューマン・コンピュータ・インターフェース技術.
二つ目は,収集したデータから推薦情報を生成し,それを目的に応じて変換する機械学習,統計的予測,そして情報検索の技術.
三つ目は,推薦に必要な情報を蓄積し,処理し,流通させる基盤技術であるデータベース,並列計算,そしてネットワーク関連の技術.
本稿では,三つ目の基盤技術については扱わず,推薦システム独自の側面が強い,ヒューマン・コンピュータ・インターフェース技術と機械学習関連技術を中心に主立った研究を紹介する.
